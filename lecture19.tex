% !TEX root = main.tex

\section{Lecture 19: Friction \& Stress in models}
\begin{flushright}\textbf{[by Craig Stewart and David Gwyther]}\end{flushright}

Intro paragraph
- need to treat frictional dissipation in models\\
- that term in NS that we kept ignoring.\\
- split into internal frictional dissipation (i.e. straight from lectures)\\
- and frictional dissipation at the vertical boundaries (application from ice/ocean BL theory that we write from scratch)\\


Fluid elements are subject to two general kinds of force; body forces, which act through the medium in proportion to mass, and contact forces, which are transmitted between adjacent regions of fluid. Body forces include gravity, electromagnetic forces, and the apparent Coriolis force. Contact forces, generated by the interaction of fluid elements with their neighbours, are represented by the second order stress tensor $\boldsymbol{T}$. These interactions between adjacent fluid elements provide the means to move momentum between regions of the fluid, and through molecular viscosity, to dissipate kinetic energy into heat, and are the essential difference between particle and continuum mechanics.  % Griffies Sec 26.1

Newton's second law, applied to a control volume $\partial V$ of fluid with density $\rho$, and baricentric velocity $v$, provides the Eularian conservation of momentum equation,\\
\begin{equation}
    \rho\frac{D\boldsymbol{v}}{Dt} = \rho\boldsymbol{f} +  \boldsymbol{\nabla}\cdot\boldsymbol{T}.
    \label{L19:GeneralFormMomentum}
\end{equation}
This simply states that the material acceleration of the fluid times its density, is equal to the sum of forces acting on the element, including the body force $\rho \boldsymbol{f}$ and the contact force $\boldsymbol{\nabla}\cdot\boldsymbol{T}$. Although the contact force is the integral of $\boldsymbol{T}$ over the fluid element surface, Gauss's law allows us to write this as the volume integral of $\boldsymbol{\nabla}\cdot\boldsymbol{T}$ over the fluid volume $\partial V$.

In three dimensional Cartesian space the stress tensor $\boldsymbol{T_{ij}}$ has 9 elements that describe the three orthogonal stresses acting on each of three planes orthogonal to the Cartesian basis vectors. 
\begin{equation}
T = 
\begin{bmatrix}
T_{xx}&T_{xy}&T_{xz}\\
T_{yx}&T_{yy}&T_{yz}\\
T_{zx}&T_{zy}&T_{zz}.\\
\end{bmatrix}
\end{equation}
Here the first index describes the direction of the stress component, and the second index describes the face normal direction.\\
%These describe the component of stress in the $i_{th}$ dimension acting on the face normal to the $j_{th}$ dimension.\\

The general stress tensor represents two fundamental types of stress which can be useful to separate. Terms on the diagonal represent stresses acting normal to the faces, and these are associated with reversible momentum transfer. Off-diagonal terms represent shear stresses that result in irreversible momentum exchange through friction.\\

Typically, to represent these physically distinct processes $\boldsymbol{T}$ is decomposed as follows:
\begin{equation}
    \boldsymbol{T} = \tau + 
    \begin{bmatrix}
P&0&0\\
0&P&0\\
0&0&P.\\
\end{bmatrix},
    \label{L19:StressDecomposition}
\end{equation}
where the scalar pressure $P$ is the mean of the normal stresses ($P = \frac{1}{3}(T_{xx} + T_{yy} + T_{zz}$)), and $\tau$ the deviatoric stress tensor. Because the pressure P is equal in the three dimensions, it acts purely to compress the fluid. In contrast, the deviatoric stress $\tau$ provides no net compression, but acts to distort the fluid element. 

While gradients in the pressure field provide a force (per unit volume) to accelerate the flow, shear stresses in $\tau$ are the result of relative motion between fluid elements, and determination of these stresses requires knowledge of the constitutive relation of the fluid, i.e. the relationship between stress and strain rate. For water, shear stresses are linearly proportional to shear strain rates, as represented by the Newtonian fluid constitutive relation,

%We begin by reminding ourselves of the form by which frictional stress $\tau_{ij}$, can be expressed in terms of the strain (e.g. shear in velocity) it produces,
\begin{align}
  \tau_{ij} &= \mu \frac{\partial v_i}{\partial x_j}
  \label{L19:StressStrainRelation}
\end{align}
where $\mu$ is the dynamic viscosity of water.

Substituting Eqs.~\eqref{L19:StressStrainRelation} and \eqref{L19:StressDecomposition} into Eq.~\eqref{L19:GeneralFormMomentum}, expanding the material derivative, including the body forces due to gravity and Coriolis acceleration, and dividing by the mean density $\rho_0$ provides the Navier-Stokes equation,
\begin{equation}
    \frac{\partial v_i}{\partial t} + v_j\frac{\partial v_i}{\partial x_j} + f\sum_{ijk}\underbrace{\delta_{j3}}_{\hat{\boldsymbol{z}}} v_k = \frac{-\nabla P}{\rho_0}-\underbrace{\delta_{i3}}_{\hat{\boldsymbol{z}}} g + \nu \frac{\partial^2 v_i}{\partial {x_j}^2}.\label{L19:NavierStokes}
\end{equation}
Here the first two terms represent the time tendency and the advective components of the material derivative of the fluid velocity. The third term represents the Coriolis acceleration with Coriolis parameter $f$ and $\delta_{j3}$ the unit normal vector in the z direction. Terms on the right hand side represent forces due to the pressure gradient, gravity, and shear stresses. Here $\nu$ is the kinematic viscosity $\nu \equiv \frac{\mu}{\rho_0}$.

\color{black}
%We consider the incompressible Navier-Stokes equation, where we now include the last term on the RHS,
%\begin{equation}
%    \frac{\partial v_i}{\partial t} + v_jv_{i,j}=f\sum_{ijk}\underbrace{\delta_{j3}}_{\hat{\boldsymbol{z}}} v_k = \frac{p_{,j}}{\rho_0}-\underbrace{\delta_{i3}}_{\hat{\boldsymbol{z}}} g + \nu v_{i,jj}.\label{L19:NavierStokes}
%\end{equation}
%In Eq.~\eqref{L19:NavierStokes}, $\nu$ is the kinematic viscosity $\nu \equiv \frac{\mu}{\rho_0}$.

\textcolor{red}{NOW ADD A FUNNY PICTURE THAT LOOKS LIKE A POTATO IF WE ARE FOLLOWING ANDY'S LECTURE.}

We can use this understanding of the frictional stress to gain some handle on internal dissipation within the ocean, specifically the time-scale for the dissipation of waves in the ocean.

\textcolor{blue}{Because we are interested in the time scale for dissipation of mortion, the relevant terms in the Navier Stokes equation are the first term, which indicates the rate of change of velocity with respect to time, and the last term which indicates the viscous forces that contribute to dissipation. While the remaining four terms are important to the instantaneous solution at any point, these can be ignored as they operate to move momentum throughout the domain, but do not contribute to the net dissipation of energy.} \textcolor{red}{Navid - does this sound ok? Is there a better justification to the cancellation of these terms?}

On this basis, the simplified momentum equation reduces to 
\begin{equation}
    \frac{\partial v_i}{\partial t} = \nu \frac{\partial^2 v_i}{\partial {x_j}^2}.\label{L19:SimplifiedNavierStokes}
\end{equation}

We now consider an arbitrary wave, this time just in the $x$-direction, which can be represented by its Fourier components
\begin{equation}
    v_1 = u(x) = \sum_{k} \hat{u}_k e^{2 \pi ikx}.\label{L19:FourierTransWave}
\end{equation}
Here the ${\hat{u}}_k$ nomenclature indicates the $k$-th Fourier component \textcolor{blue}{and k is the horizontal wave number in the x direction, where $k=\frac{2\pi}{\lambda}$ and has units [m$^{-1}$]}. \textcolor{red}{Navid - how does this sound?}

Substituting Eq.~\eqref{L19:FourierTransWave} into Eq.~\eqref{L19:SimplifiedNavierStokes}, and differentiating, we obtain
\begin{align}
    \frac{\partial v_i}{\partial t} &= \nu \frac{\partial^2 u(x)}{\partial x^2}\\
    &=\nu(2\pi i)^2\sum_k k^2 \hat{u}_k e^{2\pi ikx}\\
    &=-(2\pi)^2\nu\sum_kk^2 u_k. \label{L19:FrictionalDiss}
\end{align}
\textcolor{red}{Craig: Two issues here, would be great to get your thoughts on Navid:\\ 
(A) - why are we differentiating wrt x? I thought the point of shear stresses is that they're proportional the velocity gradient normal to the velocity (i.e. du/dy in this case? I have a suspicion that in a sinusoidal wave in incompressible fluid somehow du/dx might be the same, but this is certainly not the case for a steady shear flow...\\ 
(B) What happened to the exponential term in the last line? My guess is that we're averaging over the entire domain in which case the $e$ term takes all values around the unit circle and somehow averages to 1, but I'm lost here. \\David: With regards (B) I thought that we just recall that $\sum_{k} \hat{u}_k e^{2 \pi ikx}$ is equal to $u(x)$. But I could be wrong, and in which case the summation of $k$ and the $u_k$ should be replaced with $u(x)$. } 


Substituting this simple 1-D frictional dissipation relationship shown in Eq.~\eqref{L19:FrictionalDiss} into Eq.~\eqref{L19:SimplifiedNavierStokes}, we obtain
\begin{equation}
    \frac{\partial u_k}{\partial t} = -(2\pi)^2\nu k^2 u_k.
\end{equation}

%\textcolor{red}{Check reasoning below}\\
Now we are interested in the time required for the flow to cease due to viscous dissipation so take the finite difference form of Eq.~\eqref{L19:FrictionalDiss}, and set the velocity change $\Delta u_k$ equal to the initial wave velocity $u_k$,
\begin{equation}
    \frac{\cancel{\Delta u_k}}{\Delta t} = -(2\pi)^2\nu k^2 \cancel{u_k}
\end{equation}
to obtain a relationship showing the timescale over which dissipation through internal frictional stresses will act to damp a wave.

\begin{equation}
    \frac{1}{\Delta t} \sim 4\pi^2 \nu k^2.
\end{equation}
If we substitute values which are approximately representative for the ocean, 
\begin{equation}
    \nu \sim 10^{-6}, k \sim 10^{-4}
\end{equation}
(i.e. a wavelength of ~60 km), we can approximate the decay timescale for the ocean due to frictional dissipation,

\begin{align}
    \Delta t &\sim \frac{1}{4\times 10 \times 10^{-6} \times 10^{-8}}\\
    &= 4\times 10^{13} \mathrm{s}
\end{align}
This illustrates that the timescale for frictional dissipation of waves within the ocean \textit{in the absence of turbulence} is millions of years (1.3 million years).

\textcolor{red}{Craig: I have to say here this only applies for waves in a laminar ocean. Given that the real ocean is highly turbulent with consequently massively higher localised strain rates I'm not sure how this applies to the real world. I guess we could redo the calculation for waves of length $10^{-2}$m (this gives $k = 600$ and $\Delta t $ of 0.1s - this seems a bit fast? Thoughts?\\ Dave: I think this is a nice little note to end with. We could, as you suggest, end it with the calculation for smaller waves, order 1cm, and show how rapidly they will decay. As a result, there is a motivation to include these smaller processes. However, rapid timescales and fine processes are difficult to computationally solve efficiently, suggesting the need for sub-grid scale parameterisations. What do you think about ending it like this? Navid?}

%The next bit of notes suggested he was about to get into Reynolds decomposition, and sub-grid scale processes, presumably leading to the need for much greater values of effective viscosity in models, and due to stratification the anisotropy between horizontal and vertical diffusivity... This would be useful stuff but (a) my notes trail off here and (b) I've run out of time for this. He also continued in the next lecture with some discussion of Laplacian, Biharmonic and Smagorinski viscosity, but I can't do that at this stage... Dave: Yeah, that's fine, I can't really do it either. I think that would be covered in the next lecture anyway, as from memory this is where we stopped for a break..

%\noindent\rule{\textwidth}{1pt}
% How about we just leave the surface frcition stuff..
%Now consider the friction that arises at the boundaries of the ocean. This could include at side boundaries or horizontal boundaries such as topography (sea floor) or the ice-ocean interface beneath ice shelves or sea ice. We will consider the ice/ocean boundary and give an illustration of how the frictional stresses transfer momentum forming a boundary layer region with substantially different properties. This boundary layer region, which is often at or below vertical grid scale, will generally need to be parameterised.

%FIGURE, with layer definitions

%law of wall definition. Note that this was covered in Andy's ``Horizontal viscosity'' lecture under the heading ``Deremble et al 2011''. So we should be careful not to double up too much

%quadratic drag derivation (prob combined with LOTW definition above?). Could also do linear or log laws?

%implementation in models

%Gap between how it's implemented in models and how it is in reality! We could really go wild in this section couldn't we!
