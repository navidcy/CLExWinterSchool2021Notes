% !TEX root = main.tex

\section{Lecture 21: Advection Schemes and Numerical Diffusion}
\begin{flushright}\textbf{[by Ryan Holmes]}\end{flushright}

In this lecture we took an introductory look at numerical advection
schemes and examined some of their basic properties and issues.

Consider the one-dimensional advection equation,
\begin{equation}
\frac{\partial\Theta}{\partial t} = -u \frac{\partial\Theta}{\partial x},\label{L21:ADE}
\end{equation}
where $\Theta(x,t)$ is a tracer and $u$ is a \textit{positive}
constant. A solution can be easily written down,
\begin{equation}
\Theta(x,t) = \Theta_0(x-ut),\label{L21:ADEsol}
\end{equation}
where $\Theta_0(x)$ is the tracer distribution at the initial time
$t=0$. In other words, the initial tracer distribution should
translate toward the positive $x$ direction without changing
shape. Note that throughout this lecture we will ignore the influence
of boundary conditions on $x$ (although these can easily be
incorporated).

We now consider solving Eq. \eqref{L21:ADE} on a discrete grid using
finite differences, to see how a numerical solution may differ from
the known analytic solution Eq. \eqref{L21:ADEsol}. We discretize
Eq. \eqref{L21:ADE} onto a regularly spaced grid $x_i$
($i\in 1,\cdot\cdot\cdot N$) with grid spacing $\Delta x$ and with a
regular time step $\Delta t$ such that the discrete time vector is
given by $t^k$ ($k\in 0,\cdot\cdot\cdot M$). $\Theta_i^k$ denotes the
value of the tracer at location $x_i$ and time $t^k$.

We will first examine the \textbf{forward-time, backward-space}
discretization of Eq. \eqref{L21:ADE},
\begin{equation}
  \frac{\Theta_i^{k+1}-\Theta_i^k}{\Delta t} = -u
  \frac{\Theta_i^k-\Theta^k_{i-1}}{\Delta x}.\label{L21:FTBS}
\end{equation}
Here we have discretized the time derivative with a forward
difference and the space derivative with a backward difference. This
equation is easy to time step forward by rearranging to determine the tracer
concentration at time step $k+1$ from the tracer concentration at time
step $k$,
\begin{equation}
  \Theta_i^{k+1} = -\frac{u\Delta t}{\Delta x}\left(\Theta_i^k-\Theta^k_{i-1}\right)+\Theta_i^k.\label{L21:EXTS}
\end{equation}

How does Eq. \eqref{L21:FTBS} differ from the continuous version
Eq. \eqref{L21:ADE}? To answer this question we employ \textbf{Hirt's
  Stability Analysis}. Each term in Eq. \eqref{L21:FTBS} is written in
terms of a Taylor expansion about $\Theta_i^k$. I.e.,
\begin{align}
  \Theta_i^{k+1} &= \Theta_i^k+\Delta t\frac{\partial\Theta}{\partial
  t}\Big|_i^k+\frac{(\Delta t)^2}{2}\frac{\partial^2\Theta}{\partial
  t^2}\Big|_i^k + \mathcal{O}((\Delta t)^3), \\
  \Theta_{i-1}^k &= \Theta_i^k-\Delta x\frac{\partial\Theta}{\partial
  x}\Big|_i^k+\frac{(\Delta x)^2}{2}\frac{\partial^2\Theta}{\partial
  x^2}\Big|_i^k + \mathcal{O}((\Delta x)^3).
\end{align}
Substitution of these forms into Eq. \eqref{L21:FTBS}, followed by
some tedious rearrangement (This is left as an exercise --- note that one must use the continuous form Eq. \eqref{L21:ADE} to convert the remaining
time derivatives into spatial derivatives) yields,
\begin{equation}
  \frac{\partial\Theta}{\partial t} = -u
  \frac{\partial\Theta}{\partial x} + \frac{u}{2}\left(\Delta
    x-u\Delta t\right)\frac{\partial^2\Theta}{\partial x^2} +
  \mathcal{O}\left((\Delta x)^2,\Delta x\Delta t,(\Delta t)^2\right),\label{L21:FTBSepde}
\end{equation}
Eq. \eqref{L21:FTBSepde} is known as the \textbf{modified equivalent
  PDE for the FTBS scheme}.

Comparing this modified PDE to our original continuous PDE
[Eq. \eqref{L21:ADE}] shows that the finite difference discretization
results in a leading order error term proportional to
$\frac{\partial^2\Theta}{\partial x^2}$ - i.e. a diffusive correction
term.  We set out to solve an advection equation, but by the act of
discretizing the equation using finite differences we have in fact
ended up with an advection--diffusion equation. The coefficient in
front of the leading order error term can be thought of as a
\textbf{numerical diffusivity},
\begin{equation}
  \kappa_{num} = \frac{u}{2}\left(\Delta x-u\Delta t\right).\quad\text{ FTBS}\label{L21:FTBSKnum}
\end{equation}
If $\kappa_{num}>0$ then the diffusion is down-gradient, resulting in
the weakening of any sharp gradients in the tracer field through
\textbf{numerical diffusion}. 

However, $\kappa_{num}$ is not guaranteed to be positive. If
$\kappa_{num}$ is negative, then the leading-order correction term
drives up-gradient diffusion (or \textit{un-mixing}) which acts to
increase gradients. Since the rate at which gradients are increased
depends on the gradients themselves, this can result in exponential
growth and the solution becomes unstable. Introducing the
\textbf{Courant} number,
\begin{equation}
  C = \frac{u\Delta t}{\Delta x},
\end{equation}
the FTBS numerical diffusivity Eq. \eqref{L21:FTBSKnum} can be
rewritten as,
\begin{equation}
  \kappa_{num} = \frac{(\Delta x)^2}{2\Delta t}C(1-C).
\end{equation}
Therefore the FTBS scheme is stable if the Courant number is in the
range $0\leq C\leq 1$. Physically, this condition can be interpreted
in terms of the \textit{cone of influence} of the backward in space
operator. For the simple advection equation Eq. \eqref{L21:ADE} the
solution propagates along characteristics defined by $x=ut$. If this
information propagates across a grid cell in less than a time step
(corresponding to $C>1$), then the solution becomes unstable as the
value $\Theta_i^{k+1}$ cannot be determined from the values
$\Theta_{i-1}^k$ and $\Theta_i^k$ (if $C=1.5$ then we would need to
utilize $\Theta_{i-2}^k$ and $\Theta_{i-1}^k$, XXX: Draw a diagram to
illustrate this). 

The FTBS scheme is only stable when the velocity is positive $u>0$
(again because for $u<0$ the cone of influence of the backward in
space operator does not match the characteristics $x=ut$ of the
solution). This corresponds to an \textit{upwind} advection scheme. If
$u$ was instead negative, the forward-time, forward-space scheme
(FTFS) would be stable (providing $|C|<1$). If the sign of $u$ varies
throughout the domain one could choose to use a forward or a backward
spatial finite difference depending on the sign of $u$ (known as an
\textit{upwind} advection scheme).

What about the case $C=1$? With
$C=1$, $\kappa_{num}=0$ and in fact all of the error terms are
identically zero. This corresponds to the case where the solution
propagates exactly one grid cell with each time step, and thus
the discretized solution is exact. However, it is only
possible to have $C=1$ at every location if the velocity is constant.

What about centered differences in space? The \textit{forward time,
  centered space} (FTCS) discretization of Eq. \eqref{L21:ADE} is,
\begin{equation}
  \frac{\Theta_i^{k+1}-\Theta_i^k}{\Delta t} = -u
  \frac{\Theta_{i+1}^{k}-\Theta^k_{i-1}}{2\Delta x}.\label{L21:FTCS}
\end{equation}
The corresponding modified equivalent PDE is,
\begin{equation}
  \frac{\partial\Theta}{\partial t} = -u
  \frac{\partial\Theta}{\partial x} - \frac{u^2\Delta t}{2}\frac{\partial^2\Theta}{\partial x^2} +
  \mathcal{O}\left((\Delta x)^2,\Delta x\Delta t,(\Delta
    t)^2\right).\quad\text{ equivalent PDE for FTCS}\label{L21:FTCSepde}
\end{equation}
In this case the numerical diffusivity,
\begin{equation}
  \kappa_{num} = -\frac{u^2\Delta t}{2},
\end{equation}
is always negative and thus the FTCS is unstable for any chosen time
step. One could stabilize schemes such as this by adding explicit
diffusion to the solution with a diffusivity that exceeds the
numerical diffusivity (e.g. a term
$\kappa_p\frac{\partial^2\Theta}{\partial x^2}$ to Eq. \eqref{L21:ADE},
where $\kappa_p>\kappa_{num}$). In the FTCS case such a scheme is
known as the \textit{Lax-Wendroff} scheme.

So far we have considered only forward time-stepping schemes. These
are known as \textit{explicit} in time, and are computationally easy
to time step as they can be written in the form of
Eq. \eqref{L21:EXTS}, where all of the RHS terms are known at the
current time and thus the tracer field at the next time step is easy
to determine. An example of an \textit{implicit} scheme is the
\textit{backward time, backward space} (BTBS) scheme,
\begin{equation}
  \frac{\Theta_i^{k}-\Theta_i^{k-1}}{\Delta t} = -u
  \frac{\Theta_i^{k}-\Theta^k_{i-1}}{\Delta x}.\label{L21:BTBS}
\end{equation}
The corresponding modified equivalent PDE for BTBS is,
\begin{equation}
  \frac{\partial\Theta}{\partial t} = -u
  \frac{\partial\Theta}{\partial x} + \frac{u}{2}\left(\Delta x + u
    \Delta t\right)\frac{\partial^2\Theta}{\partial x^2} +
    \mathcal{O}\left((\Delta x)^2,\Delta x\Delta t,(\Delta
      t)^2\right).\quad\text{ equivalent PDE for BTBS}\label{L21:FTCSepde}
\end{equation}
In this case the numerical diffusivity,
\begin{equation}
  \kappa_{num} = \frac{u}{2}\left(\Delta x + u\Delta t\right),
\end{equation}
is stable for any $u$ providing that we choose a large enough time
step. I.e. there is no upper time step stability restriction (or
Courant condition) on the BTBS scheme. However, for this implicit
scheme the solution is more difficult to time step, as the RHS of
Eq. \eqref{L21:BTBS} is evaluated at time level $k$ when only time
level $k-1$ is known. To solve Eq. \eqref{L21:BTBS} for the new time
level $k$ we rewrite it as a matrix equation,
\begin{equation}
  A_{ij}\Theta_j^k = \Theta_i^{k-1},\label{L21:BTBSmat}
\end{equation}
where
$A_{ij} = (1+\frac{u\Delta t}{\Delta x})\delta_{ij} - \frac{u\Delta
  t}{\Delta x}\delta_{(i-1)j}$. Eq. \eqref{L21:BTBSmat} can then be
solved by matrix inversion, but at considerably more cost than the
simple forward time-stepping schemes considered above.

Finally, we consider the \textit{centered-time, centered-space} (CTCS)
scheme (centered-time time stepping is also known as leapfrog),
\begin{equation}
  \frac{\Theta_i^{k+1}-\Theta_i^{k-1}}{2\Delta t} = -u
  \frac{\Theta_{i+1}^{k}-\Theta^k_{i-1}}{2\Delta x}.\label{L21:CTCS}
\end{equation}
The corresponding modified equivalent PDE for CTCS is,
\begin{equation}
  \frac{\partial\Theta}{\partial t} = -u
  \frac{\partial\Theta}{\partial x} - \frac{u(\Delta x)^2}{6}(1-C^2)\frac{\partial^3\Theta}{\partial x^3} +
  \mathcal{O}\left((\Delta x)^3,(\Delta x)^2\Delta t,...\right).\quad\text{ equivalent PDE for CTCS}\label{L21:CTCSepde}
\end{equation}
This scheme is second-order accurate and the leading-order error term
now depends on the third-derivative of $\Theta$. This is known as a
\textit{dispersive} advection scheme, as the third-order derivative
causes different wave-number components of the solution to propagate
at different speeds (Exercise: write $\Theta$ in terms of its Fourier
series in $x$ and substitute into Eq. \eqref{L21:CTCSepde}, retaining only
the leading-order error term. You will find that the third-order
derivative modifies the propagation speed of the solution so that it
is no longer equal to $u$, and is dependent on
wave-number). Dispersive advection schemes can develop oscillations as
a result of this dispersion, or separation of different wave-number components
(a good example to consider is where $\Theta_0$ is a step
function). Dispersive advection schemes are often combined with
\textit{flux limiters} which prevent this development of extrema in
the tracer distribution.

For more information see \citet{Lomax2001}.
