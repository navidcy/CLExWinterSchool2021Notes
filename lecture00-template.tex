 % !TEX root = main.tex

\section*{Lecture 0: Writing guidelines and more}
\begin{flushright}\textbf{[by Navid]}\end{flushright}


\subsection*{Note-writing Guidelines}


Please try to stick to the following guidelines while writing your notes.

\begin{itemize}

\item 
Use Australian spelling.

\item
All equations should be numbered, like this one:
\begin{equation}
  \frac{\partial\rho}{\partial t} = \boldsymbol{\nabla}\cdot\boldsymbol{v}.
\end{equation}

\item
Equations should read as prose, i.e., they should be punctuated. (Have a look at this entertaining little piece by Mermin regarding combining equations in text:
\begin{quote}
  N. D. Mermin. What's wrong with these equations? \emph{Phys. Today}, \textbf{42(10)}, 9–11, 1989.\\(Available online at \href{http://www.pamitc.org/documents/mermin.pdf}{http://www.pamitc.org/documents/mermin.pdf}.)
\end{quote}

\item
You can colour text \textcolor{red}{like this} using \verb"\textcolor{color}{text}". Please resist overdoing it though.

\item
Vectors are denoted with \textbf{bold} using \verb"\boldsymbol{...}".

\item
Notation is as follows:
\begin{align}
 \text{horizontal velocity field:}\quad &\boldsymbol{u} = (u, v), \\
 \text{three-dimensional velocity field:}\quad &\boldsymbol{v} = (u, v, w),\\
 &\ \ \vdots \nonumber
\end{align}

\item
Let's use $\times$ for cross products, e.g., $[\boldsymbol{a}\times\boldsymbol{b}]_i = \epsilon_{ijk}a_jb_k$.

\item
Since these are notes, it might be useful to cross out term in equations as, e.g., 
\begin{equation}
    \cancel{x} + 5 + y  = \cancel{x}.
\end{equation}

\item
Fourier transformed variables should be denoted with hats:
\begin{equation}
    \boldsymbol{u}(\boldsymbol{x}) = \sum_{\boldsymbol{k}} \hat{\boldsymbol{u}}_{\boldsymbol{k}} e^{i \boldsymbol{x}\cdot\boldsymbol{k}}.
\end{equation}

\item
Partial derivatives should be either written in full, e.g., $\partial\rho/\partial t$ or as $\partial_t\rho$ but \emph{\bf not} using $\rho_t$ notation.

\item
Andy used in places subscript-comma notation to denote partial derivatives, e.g., $\partial_y u = u_{,2}$. Since this notation was not widely spread throughout the lectures, let's just stick to the usual $\partial_y u$.

\item
Add figure files in the \verb"figures/" folder and then incorporate them in the text using the usual 
\begin{quote}
    \verb"\begin{figure}"\\
    ...\\
    \verb"\end{figure}"
\end{quote}
environment with \verb"\includegraphics[width=3in]{figurename}". Note that you need not include the figure's folder path, e.g., \verb"\includegraphics[width=3in]{figures/figurename}".

\item
Hand-drawn figs/diagrams are mostly welcome!

\item
Refrain from using $\cdot$ or $\times$ for plain multiplication (except when it's absolutely necessary to avoid ambiguity)! Symbol $\cdot$ (\verb"\cdot") should only denote a dot product, while $\times$ (\verb"\times") should only denote a cross product.

\end{itemize}

\begin{figure}[h!]
  \centering
  \includegraphics[width=3in]{newyorkercomic}
  \caption{A caption is often useful.}
\end{figure}

